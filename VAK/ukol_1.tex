\documentclass{article}

\usepackage{amssymb}

\title{BI-VAK - úkol 1}
\author{Patrik Drbal}

\begin{document}
\maketitle

\textbf {\\Zadání 8\\}
Matematickou indukcí dokažte, že pro každé $n>1$ platí následující rovnost.
\[ \sum_{i=1}^{n} i(i+1)(i+2) = \frac{n(n+1)(n+2)(n+3)}{4} \]

\textbf{Základní krok pro $n=2$:}
\[ \sum_{i=1}^{2} i(i+1)(i+2) = 1\cdot2\cdot3 + 2\cdot3\cdot4 = 30 \]
\[ \frac{2(2+1)(2+2)(2+3)}{4} = 2\cdot3\cdot5 = 30 \]
Pro $n=2$ rovnost platí.

\bigbreak
\textbf{Indukční krok pro $n>=2$:}
\[
\sum_{i=1}^{n} i(i+1)(i+2) = \frac{n(n+1)(n+2)(n+3)}{4} \Rightarrow
\sum_{i=1}^{n+1} i(i+1)(i+2) = \frac{(n+1)(n+2)(n+3)(n+4)}{4}
\]

Upravíme sumu
\[
\sum_{i=1}^{n+1} i(i+1)(i+2) =
(n+1)(n+2)(n+3) + \sum_{i=1}^{n} i(i+1)(i+2)
\]

Z indukčního předpokladu je tento výraz roven výrazu
\[ (n+1)(n+2)(n+3) + \frac{n(n+1)(n+2)(n+3)}{4} \]

Převedeme na společný jmenovatel
\[ \frac{ n(n+1)(n+2)(n+3) + 4(n+1)(n+2)(n+3) }{4} \]

A vytkneme $(n+1)(n+2)(n+3)$ a díky komutativitě násobení přesuneme doleva
\[ \frac{ (n+1)(n+2)(n+3)(n+4) }{4} \]

Postupnými ekvivalentními úpravami sumy na levé straně jsme získali výraz na pravé straně, čili indukční krok platí pro všechna $n>=2$.

\bigbreak
Rovnost, kterou jsme měli dokázat, je tedy korektní. $\blacksquare$


\textbf {\\Bonusové zadání\\}
\[ \frac{m}{n} - \frac{1}{\lceil \frac{n}{m} \rceil} \]

Převedeme na společný jmenovatel
\[ \frac{ m \cdot \lceil \frac{n}{m} \rceil - n }{ n \cdot \lceil \frac{n}{m} \rceil } \]
Výraz $m \cdot \lceil \frac{n}{m} \rceil$ je nanejvýš $n+m$, protože $\lceil \frac{n}{m} \rceil = \frac{k}{m}$, kde $k >= n$ je nejbližší takové, že $m \mid k$. Čili $m \cdot \frac{k}{m} = k$, kde $k$ je nanejvýš $n+m$.

Tedy výraz výše je roven výrazu
\[
\frac{ n+m-n }{ n \cdot \lceil \frac{n}{m} \rceil } =
\frac{ m }{ n \cdot \lceil \frac{n}{m} \rceil }
\]

A protože $\lceil \frac{n}{m} \rceil >= 2$ (plyne z předpokladu $1<=m<n$), platí
\[
\frac{ m }{ n \cdot \lceil \frac{n}{m} \rceil } <
\frac{m}{n}
\]

Tedy po každé iteraci algoritmu se hodnota výrazu zmenší, a protože pracujeme pouze s kladnými celočíselnými čitateli a jmenovateli a každá iterace trvá konečný počet kroků, po konečném počtu iterací, které budou trvat konečný čas, se hodnota výrazu zmenší na nulu a v tomto okamžiku se algoritmus zastaví. $\blacksquare$

\end{document}