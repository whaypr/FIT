\documentclass{article}

\usepackage{amsmath, amsthm, amssymb}

\title{BI-VAK - úkol 9}
\author{Patrik Drbal}

\begin{document}
\maketitle

\textbf {\\Zadání 1 - Maximální řez\\}

\begin{center}
Vstup
\end{center}
Graf $G$.

\begin{center}
Výstup
\end{center}
Rozdělení vrcholů na množiny $A$ a $B$ takové, že počet hran mezi $A$ a $B$ je maximalizován.

\textbf {\\Řešení\\}

Za každý vrchol $v_i \in V(G)$ zvolíme logickou proměnnou $x_i$.

Hodnota proměnné $x_i$ bude indikovat, jestli vrchol patří do komponenty $A$ ($1$), nebo do komponenty $B$ ($0$).

\vspace{5mm}
Za každou hranu $e = \{u,v\} \in E(G)$ zvolíme logickou proměnnou $y_{u,v}$.

Hodnota proměnné $y_{u,v}$ bude indikovat, jestli hrana patří ($1$) do maximálního řezu, nebo nepatří ($0$).

Počet proměnných $y_{u,v}$ nastavených na $1$ chceme maximalizovat.

\vspace{5mm}
\hrule\vspace{5mm}
Nejdříve potřebujeme pro každou dvojici vrcholů $x_i$, $x_j$ zajistit, aby ve výsledném řezu nebyla hrana $y_{i,j}$, pokud jsou oba ve stejné komponentě (mají stejnou hodnotu).

\vspace{5mm}
Máme tedy
\[ (x_i \land x_j) \implies \neg y_{i,j} \]
pro $x_i$ a $x_j$ v komponentě $1$.

A
\[ (\neg x_i \land  \neg x_j) \implies \neg y_{i,j} \]
pro $x_i$ a $x_j$ v komponentě $0$

Dohromady dostáváme výraz
\[ (x_i \land x_j) \vee (\neg x_i \land  \neg x_j) \implies \neg y_{i,j} \]

\vspace{5mm}
Zároveň chceme zajistit, že pokud hrana $y_{i,j}$ není vybrána do maximálního řezu, vrcholy $x_i$ a $x_j$ jsou ve stejných komponentách. Čili spolu s předchozím dostáváme následující ekvivalenci
\[ (x_i \land x_j) \vee (\neg x_i \land  \neg x_j) \iff \neg y_{i,j} \]

Nebo logicky ekvivalentně
\[ \boxed{ y_{i,j} \iff (x_i \land \neg x_j) \vee ( \neg x_i \land x_j) } \]

\vspace{5mm}
\vspace{5mm}
Pro použití v SAT řešiči musíme formuli převést do CNF.

\vspace{5mm}
Přepisem formule podle pravidla
\[(a \iff b) \vdash ( (a \implies b) \wedge (b \implies a) ) \vdash ( (a \vee \neg b) \wedge (\neg a \vee b) )\]
dostáváme
\[
( y_{i,j} \vee \neg ((x_i \land \neg x_j) \vee ( \neg x_i \land x_j)) )
\wedge
( \neg y_{i,j} \vee (x_i \land \neg x_j) \vee ( \neg x_i \land x_j) )
\]

\vspace{5mm}
Což lze dalšími logicky ekvivalentními úpravami zjednodušit na 
\[ \boxed{
( y_{i,j} \vee \neg x_i \vee x_j ) \wedge
( y_{i,j} \vee x_i \vee \neg x_j ) \wedge
( \neg y_{i,j} \vee x_i \vee x_j ) \wedge
( \neg y_{i,j} \vee \neg x_i \vee \neg x_j )
} \]
A to už je formule v konjunktivním normálním tvaru.

\vspace{5mm}
\hrule\vspace{5mm}

Dále je zřejmé, že každý vrchol připojený nějakými hranami, musí alespoň jednu z těchto hran využít v maximálním řezu.

To lze dokázat rozborem případů. Předpokládáme, že žádná hrana incidentní s daným vrcholem není součástí maximálního řezu (její logická proměnná má hodnotu $0$).

\vspace{5mm}
\textbf {1)}
Vrchol je BÚNO v komponentě $1$ a všechny jeho hrany vedou do komponenty $0$.

Tato situace nikdy nenastane, protože je již zakázána pravidlem, že pokud hrana spojuje vrcholy z různých komponent, musí být součástí maximálního řezu.

I kdyby ale nastala, můžeme všechny incidentní hrany vybrat do maximálního řezu.

\vspace{5mm}
\textbf {2)}
Vrchol je BÚNO v komponentě $1$ a všechny jeho hrany vedou do komponenty $1$.

V této situaci můžeme vrchol přesunout do komponenty $0$ a tím problém převést na předchozí možnost.

\vspace{5mm}
\textbf {3)}
Vrchol je BÚNO v komponentě $1$ a některé jeho hrany vedou do komponenty $1$, zatímco ostatní jeho hrany vedou do komponenty $0$.

Všechny hrany vedoucí do komponenty $0$ lze prohlásit za hrany maximálního řezu.

\vspace{5mm}
\vspace{5mm}
Tedy pro každý vrchol $x_i$ existuje alespoň jedna hrana $y_{i,j}$, která je součástí maximálního řezu (má hodnotu $1$).

\vspace{5mm}
Toto chování lze vynutit klauzulemi
\[\boxed{ y_{i,j_1} \vee y_{i,j_2} \vee \ldots \vee  y_{i,j_n}  }\]
pro každý vrchol $x_i$

\vspace{5mm}
\hrule\vspace{5mm}
Protože chceme získat explicitní ohodnocení pro všechny proměnné $x_i$, přidáme SAT řešiči následující podmínku pro každou z těchto proměnných.
Tím ho přinutíme vybrat si jednu z hodnot.
\[\boxed{ x_i \vee \neg x_i }\]

\vspace{5mm}
\hrule\vspace{5mm}
V této fázi již máme pomocí formulí v CNF formě definované všechny podmínky pro to, aby nám SAT řešič našel některý z řezů na zadaném grafu $G$.

\vspace{5mm}
\vspace{5mm}
Abychom našli ten maximální, použijeme postup uvedený v přednášce a budeme si počítat TRUE proměnné.

\vspace{5mm}
Konkrétně budeme počítat počet proměnných $y_{i,j}$ ohodnocených jako $1$, čili vytvoříme pomyslnou tabulku nových proměnných $s_{k,l}$, kde $k,l \in \{1,2,...,|E(G)|\}$ podle přednášky.
Hrany grafu $G$ si můžeme nejdříve seřadit a očíslovat, abychom mohli tabulku využít stejně jako na přednášce.

\vspace{5mm}
Dle tohoto principu dokážeme vynutit, aby se počet hran hledaného řezu rovnal konstantě $K$ a pomocí binárního půlení pomocí $O(\log |E(G)|)$ volání SAT řešiče najít i hledaný maximální řez.
$\blacksquare$

\end{document}